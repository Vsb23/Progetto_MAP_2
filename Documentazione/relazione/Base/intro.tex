\section{Introduzione}

Il progetto in questione utilizza l'algoritmo \textbf{Quality Treshold} per l'analisi dei dati.

Tali dati possono essere estratti da un file, oppure possono essere estrati dalle tabelle di un database MySQL. 

L'algoritmo quality treshold è un algoritmo di \underline{clustering} che si basa sulla \underline{minimizzazione} della \underline{distanza tra i punti} e il centroide del cluster. 

L'algoritmo è stato sviluppato per essere utilizzato in contesti in cui i dati sono rumorosi e non lineari, e si basa su un approccio gerarchico per la creazione dei cluster.

Rispetto ad altri algoritmi di partizionamento dei dati, come il k-means, richeide una potenza di calcolo maggiore, ma a differenza di altri, non richiede di specificare il numero di cluster da creare, ma solo il raggio massimo. Inoltre, l'algoritmo è in grado di restituire sempre lo stesso risultato, nonostante esso venga eseguito ripetutamente, a differenza di altri algoritmi che possono restituire risultati diversi a seconda della loro inizializzazione.