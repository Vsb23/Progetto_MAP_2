\section{Introduzione}


Il progetto in questione utilizza l'algoritmo \textbf{Quality Threshold (QT)} per l'analisi dei dati. Tali dati possono essere estratti da un file o dalle tabelle di un database MySQL.

\section{Cos'è l'Algoritmo Quality Threshold}

L'algoritmo Quality Threshold è un algoritmo di \textbf{clustering} deterministico che appartiene alla famiglia degli algoritmi di raggruppamento gerarchico. A differenza degli algoritmi di partizionamento tradizionali, il QT non richiede di specificare a priori il numero di cluster da creare, ma si basa su un parametro di soglia di qualità (\textit{threshold}) che definisce il raggio massimo consentito per ogni cluster.

\section{Principio di Funzionamento}

L'algoritmo si basa sulla \textbf{minimizzazione della distanza} tra i punti dati e il centroide del cluster di appartenenza. Il processo di clustering avviene attraverso i seguenti passaggi:

\begin{enumerate}
\item \textbf{Selezione del punto candidato}: Per ogni punto del dataset, viene valutata la possibilità di creare un nuovo cluster
\item \textbf{Formazione del cluster}: Vengono raggruppati tutti i punti che si trovano entro il raggio di soglia specificato
\item \textbf{Selezione del cluster ottimale}: Viene scelto il cluster che contiene il maggior numero di punti
\item \textbf{Iterazione}: Il processo si ripete sui punti rimanenti fino a quando tutti i punti sono stati assegnati a un cluster
\end{enumerate}

\section{Caratteristiche Distintive}

L'algoritmo Quality Threshold presenta diverse caratteristiche che lo distinguono da altri approcci di clustering:

\subsection{Vantaggi}

\begin{itemize}
\item \textbf{Determinismo}: L'algoritmo restituisce sempre lo stesso risultato quando viene eseguito ripetutamente sullo stesso dataset, a differenza di algoritmi come il \textit{k-means} che possono produrre risultati diversi a seconda dell'inizializzazione casuale
\item \textbf{Numero di cluster automatico}: Non richiede di specificare il numero di cluster da creare, ma solo il raggio massimo (\textit{threshold}), rendendo l'algoritmo più flessibile per dataset con strutture sconosciute
\item \textbf{Robustezza al rumore}: È stato progettato per gestire efficacemente dati rumorosi e con distribuzioni non lineari
\item \textbf{Approccio gerarchico}: Utilizza una strategia gerarchica per la creazione dei cluster, garantendo una maggiore stabilità dei risultati
\end{itemize}

\subsection{Svantaggi}

\begin{itemize}
\item \textbf{Complessità computazionale}: Richiede una potenza di calcolo maggiore rispetto ad algoritmi di partizionamento come il \textit{k-means}, con una complessità temporale che può essere significativamente elevata per dataset di grandi dimensioni
\item \textbf{Sensibilità al parametro threshold}: La scelta del raggio di soglia influenza notevolmente la qualità del clustering ottenuto
\end{itemize}

\section{Ambiti di Applicazione}

L'algoritmo Quality Threshold è particolarmente adatto per contesti in cui:

\begin{itemize}
\item I dati presentano rumore significativo
\item La distribuzione dei dati è non lineare o presenta forme complesse
\item Non si conosce a priori il numero ottimale di cluster
\item È richiesta riproducibilità dei risultati
\item La qualità del clustering è più importante dell'efficienza computazionale
\end{itemize}