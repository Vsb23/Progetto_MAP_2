\section{Eseguire il progetto esteso}

\subsection{Modifiche implementative}

Rispetto alla versione base,  la funzione del client è stata completamente stravolta. Nel progetto base, infatti, il client non era altro che un'estensione del server, che si occupava di inviare richieste ad esso e di stampare le risposte. 

Ora, invece, tale funzione è stata delegata direttamente ad un bot telegram, il quale si occupa, come un vero e proprio client, di inviare le richieste al server e di stampare le risposte.


\subsection{Premessa}

Rispetto al progetto base, si è reso necessario l'utilizzo di un gestore di dipendene, in modo da poter installare le librerie necessarie per il progetto. 

\subsubsection*{Maven}

Il gestore di dipendenze scelto per questo scopo è \textbf{Maven}, il quale permette di scaricare le librerie necessarie e di gestire le versioni delle stesse.

Il vantaggio principale rispetto ad altri gestori di dipendenze, come il più blasonato Gradle, è la sua semplicità di utilizzo e la sua integrazione con gli IDE più diffusi. 

Inoltre, al contrario di Gradle, Maven non richiede la creazione di un file di configurazione complesso, ma si basa su un semplice file \texttt{pom.xml} che contiene le informazioni sulle dipendenze del progetto. 

Non richiede quindi la creazione di file di build o di scaricare i file delle dipendenze manualmente, ma il tutto viene gestito automaticamente. 

\subsubsection*{Telegram API}

Si è reso necessario l'utilizzo di un gestore delle dipendenze per via dell'utilizzo delle librerie \textbf{org.telegram.longpolling} e \textbf{org.telegram.telegrambots}, le quali permettono di interagire con l'API di Telegram. 

