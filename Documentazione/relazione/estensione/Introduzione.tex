\section{Eseguire il progetto esteso}

\subsection{Premessa}

Rispetto al progetto base, si è reso necessario l'utilizzo di un gestore di dipendene, in modo da poter installare le librerie necessarie per il progetto. 

\subsubsection*{Maven}

Il gestore di dipendenze scelto per questo scopo è \textbf{Maven}, il quale permette di scaricare le librerie necessarie e di gestire le versioni delle stesse.

Il vantaggio principale rispetto ad altri gestori di dipendenze, come il più blasonato Gradle, è la sua semplicità di utilizzo e la sua integrazione con gli IDE più diffusi. 

Inoltre, al contrario di Gradle, Maven non richiede la creazione di un file di configurazione complesso, ma si basa su un semplice file \texttt{pom.xml} che contiene le informazioni sulle dipendenze del progetto. 

Non richiede quindi òa creazione di file di build o di scaricare i file delle dipendenze manualmente. 

\subsubsection*{Telegram API}

Si è reso necessario l'utilizzo di un gestore delle dipendenze per via dell'utilizzo delle librerie \textbf{org.telegram.longpolling} e \textbf{org.telegram.telegrambots}, le quali permettono di interagire con l'API di Telegram. 

\subsubsection{Implementazioni aggiuntive}

Come accennato nella premessa, 



