\section{Eseguire il progetto esteso}

\subsection{Premessa}

Rispetto al progetto base, si è reso necessario l'utilizzo di un gestore di dipendene, in modo da poter installare le librerie necessarie per il progetto. 

\subsubsection*{Maven}

Il gestore di dipendenze scelto per questo scopo è \textbf{Maven}, il quale permette di scaricare le librerie necessarie e di gestire le versioni delle stesse.

Il vantaggio principale rispetto ad altri gestori di dipendenze, come il più blasonato Gradle, è la sua semplicità di utilizzo e la sua integrazione con gli IDE più diffusi. 

Inoltre, al contrario di Gradle, Maven non richiede la creazione di un file di configurazione complesso, ma si basa su un semplice file \texttt{pom.xml} che contiene le informazioni sulle dipendenze del progetto. 

Non richiede quindi la creazione di file di build o di scaricare i file delle dipendenze manualmente, ma il tutto viene gestito automaticamente. 

\subsubsection*{Telegram API}

Si è reso necessario l'utilizzo di un gestore delle dipendenze per via dell'utilizzo delle librerie \textbf{org.telegram.longpolling} e \textbf{org.telegram.telegrambots}, le quali permettono di interagire con l'API di Telegram. 

\subsection{Implementazioni aggiuntive}

Come accennato nella premessa, è stato implementato un bot Telegram. Tale bot permette di interagire con il progetto direttamente dall'interafccia di Telegram, rendendolo di fatto un client. 

L'adozione di Maven per la gestione delle dipendenze, inoltre ha reso inutile l'utilizzo del file \texttt{risorse.bat}, in quanto le librerie vengono scaricate automaticamente e non è più necessario gestire manualmente le versioni delle stesse.

Di per se il funzionamento del progetto non è cambiato, è stato semplicemente trasformato il modo di interazione tra esso e l'utente, trasferendola quindi da un'interfaccia cli a una più interattiva.  



