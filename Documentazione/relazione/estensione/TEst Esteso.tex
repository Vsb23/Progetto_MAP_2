\section{Test progetto esteso}

Durante la fase di test del progetto esteso, sono stati eseguiti diversi test per verificare il corretto funzionamento delle varie funzionalità dei comandi implmentanti. 
Tali test hanno coinvolto principalmente la correttezza della gestione delle varie situaioni anomale che possono incorrere durante l'esecuzione dei comandi, come ad esempio la gestione di file non trovati, errori di sintassi o problemi di accesso ai file.

Chiaramente queste situazioni anomale devono essere notificate al livello di client, per permettere all'utente di comprendere l'entità del problema e prendere le opportune misure.

Si fa notare che i test elencati sono stati eseguiti tenendo conto il normale flusso di lavoro del bot, nel senso che è stato seguito il flusso di interazione del bot per via sequenziale, senza saltare passaggi o inviare comandi in modo casuale.

\subsection{Tabelle non trovate}

Nel caso in cui venga inserito il nome di una tabella inesistente, il bot risponderà specificando che  la tabella isnerita non esiste e invita a inserire un nome di tabella valido.

\image{images/tabella inesistente.png}{comportamento tabella inesistente}{tabne}

Chiaramente l'algoritmo QT non potrà essere eseguito su una tabella inesistente, quindi il bot risponderà con un messaggio di errore specificando che non è possibile eseguire l'algoritmo QT se prima non viene caricata una tabella valida.

\image{images/esegui prima qt.png}{inconsistenza algoritmo QT}{inconsistenza}

\subsection{raggio non valido}

Nel caso si inserisca un raggio non valido, che include quindi le casistiche in cui il raggio è negativo o non numerico, il bot risponderà specificando che il raggio inserito non è valido, in quanto genererebbe un solo cluster, e pertanto  invita a inserire un raggio valido.

\image{images/raggio invalido.png}{Raggio invalido}{raggoinv}

\subsection{File non trovato}

Nel caso in cui, durante l'esecuzione del comando \texttt{Carica risultati}, il file specificato non dovesse essere presente enlla cartella \texttt{results}, il bot risponderà specificando che il file non è stato trovato. 

\image{images/file not found.png}{Esempio di file non trovato}{fnf}

Sarà quindi necessario eseguire nuovamente il comando ed inserire il nome di un file valido. 
